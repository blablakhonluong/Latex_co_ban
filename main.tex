\documentclass[12pt,a4paper,openany ]{book}
\usepackage{fouriernc}
\usepackage[utf8]{inputenc}
\usepackage{vntex}
\usepackage{amsfonts,amsmath,amssymb}
\usepackage{mathrsfs}
\usepackage{graphics}
\usepackage{pgf,tikz}
\usetikzlibrary{arrows}
\usepackage[left=2cm,right=1.5cm,top=2cm,bottom=2cm]{geometry}
\usepackage{fontawesome}
\usepackage{fancyhdr}
\usepackage{tcolorbox}
\usepackage{cases}
\newtcolorbox{exbox}{arc=0pt, outer arc=0pt, boxrule=1pt, colframe=blue}
\pagestyle{fancy}
\fancyhf{}
\lhead{\textsf{Một bài viết về hình học}}
\rhead{\textsf{Author}}
\cfoot{\textsf{\thepage}}
\rfoot{\textsf{08/2021}}
\renewcommand{\baselinestretch}{1.25}
\usepackage{wrapfig}
\setlength{\parindent}{2pt}

\begin{document}


$\dfrac{a}{b}$ - phân số cỡ to

$\frac{c}{d}$ - phân số cỡ nhỏ 

\textbf{Muốn bôi đậm: Bôi đen đoạn văn bản rồi nhấn tổ hợp Ctrl+B.}

\textit{Muốn in nghiêng: Bôi đen đoạn văn bản cần in nghiên rồi nhấn tổ hợp Ctrl+I.}

$a^b$: a mũ b

$\angle ABC$: góc $ABC$.

$\widehat{ABC}$: góc $ABC$. 

$\overset{\huge\frown}{PQ}$: cung $PQ$. 

$ a\parallel b$: đường thẳng $a$ song song đường thẳng $b$. 

$\bigtriangleup ABC$: tam giác $ABC$. 

$a\perp b$: đường thẳng $a$ vuông góc đường thẳng $b$. 

$90^\circ$: góc 90 độ. 







              
\end{document}